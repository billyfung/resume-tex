%%%%%%%%%%%%%%%%%%%%%%%%%%%%%%%%%%%%%%%%%%%%%%%%%%%%%%%%%%%%%%%%%%%%%%%%
%%%%%%%%%%%%%%%%%%%%%% Simple LaTeX CV Template %%%%%%%%%%%%%%%%%%%%%%%%
%%%%%%%%%%%%%%%%%%%%%%%%%%%%%%%%%%%%%%%%%%%%%%%%%%%%%%%%%%%%%%%%%%%%%%%%

%%%%%%%%%%%%%%%%%%%%%%%%%%%%%%%%%%%%%%%%%%%%%%%%%%%%%%%%%%%%%%%%%%%%%%%%
%% NOTE: If you find that it says                                     %%
%%                                                                    %%
%%                           1 of ??                                  %%
%%                                                                    %%
%% at the bottom of your first page, this means that the AUX file     %%
%% was not available when you ran LaTeX on this source. Simply RERUN  %%
%% LaTeX to get the ``??'' replaced with the number of the last page  %%
%% of the document. The AUX file will be generated on the first run   %%
%% of LaTeX and used on the second run to fill in all of the          %%
%% references.                                                        %%
%%%%%%%%%%%%%%%%%%%%%%%%%%%%%%%%%%%%%%%%%%%%%%%%%%%%%%%%%%%%%%%%%%%%%%%%

%%%%%%%%%%%%%%%%%%%%%%%%%%%% Document Setup %%%%%%%%%%%%%%%%%%%%%%%%%%%%

% Don't like 10pt? Try 11pt or 12pt
\documentclass[10pt]{article}

% This is a helpful package that puts math inside length specifications
\usepackage{calc}

% Layout: Puts the section titles on left side of page
\reversemarginpar

%
%         PAPER SIZE, PAGE NUMBER, AND DOCUMENT LAYOUT NOTES:
%
% The next \usepackage line changes the layout for CV style section
% headings as marginal notes. It also sets up the paper size as either
% letter or A4. By default, letter was used. If A4 paper is desired,
% comment out the letterpaper lines and uncomment the a4paper lines.
%
% As you can see, the margin widths and section title widths can be
% easily adjusted.
%
% ALSO: Notice that the includefoot option can be commented OUT in order
% to put the PAGE NUMBER *IN* the bottom margin. This will make the
% effective text area larger.
%
% IF YOU WISH TO REMOVE THE ``of LASTPAGE'' next to each page number,
% see the note about the +LP and -LP lines below. Comment out the +LP
% and uncomment the -LP.
%
% IF YOU WISH TO REMOVE PAGE NUMBERS, be sure that the includefoot line
% is uncommented and ALSO uncomment the \pagestyle{empty} a few lines
% below.
%

%% Use these lines for letter-sized paper
\usepackage[paper=letterpaper,
            %includefoot, % Uncomment to put page number above margin
            marginparwidth=1.2in,     % Length of section titles
            marginparsep=.05in,       % Space between titles and text
            margin=1in,               % 1 inch margins
            includemp]{geometry}

%% Use these lines for A4-sized paper
%\usepackage[paper=a4paper,
%            %includefoot, % Uncomment to put page number above margin
%            marginparwidth=30.5mm,    % Length of section titles
%            marginparsep=1.5mm,       % Space between titles and text
%            margin=25mm,              % 25mm margins
%            includemp]{geometry}

%% More layout: Get rid of indenting throughout entire document
\setlength{\parindent}{0in}

%% This gives us fun enumeration environments. compactitem will be nice.
\usepackage{paralist}

%% Reference the last page in the page number
%
% NOTE: comment the +LP line and uncomment the -LP line to have page
%       numbers without the ``of ##'' last page reference)
%
% NOTE: uncomment the \pagestyle{empty} line to get rid of all page
%       numbers (make sure includefoot is commented out above)
%
\usepackage{fancyhdr,lastpage}
\pagestyle{fancy}
\pagestyle{empty}      % Uncomment this to get rid of page numbers
\fancyhf{}\renewcommand{\headrulewidth}{0pt}
\fancyfootoffset{\marginparsep+\marginparwidth}
\newlength{\footpageshift}
\setlength{\footpageshift}
          {0.5\textwidth+0.5\marginparsep+0.5\marginparwidth-2in}
\lfoot{\hspace{\footpageshift}%
       \parbox{4in}{\, \hfill %
                    \arabic{page} of \protect\pageref*{LastPage} % +LP
%                    \arabic{page}                               % -LP
                    \hfill \,}}

% Finally, give us PDF bookmarks
\usepackage{color,hyperref}
\definecolor{darkblue}{rgb}{0.0,0.0,0.3}
\hypersetup{colorlinks,breaklinks,
            linkcolor=darkblue,urlcolor=darkblue,
            anchorcolor=darkblue,citecolor=darkblue}

%%%%%%%%%%%%%%%%%%%%%%%% End Document Setup %%%%%%%%%%%%%%%%%%%%%%%%%%%%


%%%%%%%%%%%%%%%%%%%%%%%%%%% Helper Commands %%%%%%%%%%%%%%%%%%%%%%%%%%%%

% The title (name) with a horizontal rule under it
%
% Usage: \makeheading{name}
%
% Place at top of document. It should be the first thing.
\newcommand{\makeheading}[1]%
        {\hspace*{-\marginparsep minus \marginparwidth}%
         \begin{minipage}[t]{\textwidth+\marginparwidth+\marginparsep}%
                {\large \bfseries #1}\\[-0.15\baselineskip]%
                 \rule{\columnwidth}{1pt}%
         \end{minipage}}

% The section headings
%
% Usage: \section{section name}
%
% Follow this section IMMEDIATELY with the first line of the section
% text. Do not put whitespace in between. That is, do this:
%
%       \section{My Information}
%       Here is my information.
%
% and NOT this:
%
%       \section{My Information}
%
%       Here is my information.
%
% Otherwise the top of the section header will not line up with the top
% of the section. Of course, using a single comment character (%) on
% empty lines allows for the function of the first example with the
% readability of the second example.
\renewcommand{\section}[2]%
        {\pagebreak[2]\vspace{1.3\baselineskip}%
         \phantomsection\addcontentsline{toc}{section}{#1}%
         \hspace{0in}%
         \marginpar{
         \raggedright \scshape #1}#2}

% An itemize-style list with lots of space between items
\newenvironment{outerlist}[1][\enskip\textperiodcentered
]%
        {\begin{itemize}[#1]}{\end{itemize}%
         \vspace{-.6\baselineskip}}

% An environment IDENTICAL to outerlist that has better pre-list spacing
% when used as the first thing in a \section
\newenvironment{lonelist}[1][\enskip\textbullet]%
        {\vspace{-\baselineskip}\begin{list}{#1}{%
        \setlength{\partopsep}{0pt}%
        \setlength{\topsep}{0pt}}}
        {\end{list}\vspace{-.6\baselineskip}}

% An itemize-style list with little space between items
\newenvironment{innerlist}[1][\enskip\textperiodcentered
]%
        {\begin{compactitem}[#1]}{\end{compactitem}}

% To add some paragraph space between lines.
% This also tells LaTeX to preferably break a page on one of these gaps
% if there is a needed pagebreak nearby.
\newcommand{\blankline}{\quad\pagebreak[2]}

%%%%%%%%%%%%%%%%%%%%%%%% End Helper Commands %%%%%%%%%%%%%%%%%%%%%%%%%%%

%%%%%%%%%%%%%%%%%%%%%%%%% Begin CV Document %%%%%%%%%%%%%%%%%%%%%%%%%%%%

\begin{document}
\makeheading{Billy Fung}


\section{Contact}
%
% NOTE: Mind where the & separators and \\ breaks are in the following
%       table.
%
% ALSO: \rcollength is the width of the right column of the table
%       (adjust it to your liking; default is 1.85in).
%
\newlength{\rcollength}\setlength{\rcollength}{1.85in}%
%
\begin{tabular}[t]{@{}p{\textwidth-\rcollength}p{\rcollength}}
\href{mailto:billy@billyfung.com}{billy@billyfung.com}\\

\end{tabular}


\section{Education}
%
\href{http://www.ubc.ca/}{\textbf{University of British Columbia}},
Vancouver, BC, Canada
\hfill  \textbf{2015}
\begin{outerlist}

\item[] B.A.Sc. Engineering Physics, Mechatronics specialization
\end{outerlist}

\section{Skills}
Python, Javascript, R, PostgreSQL, Docker, Ansible, AWS, GCP, GAMS

%\section{Hobbies}
%Rock climbing, hiking, bicycling, and photography

\section{Experience}
\href{http://billyfung.com}{\textbf{Software Consultant}}, Worldwide.
\begin{outerlist}
  \item[]
	\hfill  \textbf{February 2016 to Present}
\begin{innerlist}
	\item specialising in backend software development, assisting clients to build up data-driven web applications
    \item building and supporting Python based web services for a peer to peer electricity retailer
    \item developing trading and data analytics platform for wholesale electricity trading
    \item using machine learning to build predictive models around electricity demand and consumption
    \item managing and maintaining cloud based servers, namely within AWS and GCP
    \item design, implementation, and deployment of mathematical optimisation models
\end{innerlist}
\end{outerlist}

\blankline

\href{http://billyfung.com/dve/}{\textbf{DVE Technologies}}, Vancouver, Canada.
\begin{outerlist}
	\item[] \textit{Lead Engineer / Cofounder}
	\hfill  \textbf{December 2014 to November 2015}
\begin{innerlist}
	\item created and developed the company website, along with maintaining the cloud systems
    \item developed and tested electrical circuits for the prototype device
    \item aided in refining and optimising of software algorithms to fine tune head tilt gesture detection
    \item used Solidworks to design the physical enclosure for 3d printing and for plastic injection mold
\end{innerlist}
\end{outerlist}

\blankline


{\textbf{Blackberry Limited}}, Ontario, Canada.
\begin{outerlist}
	\item[] \textit{Reliability Systems Development Intern}
	\hfill  \textbf{May 2011 to August 2012}
\begin{innerlist}
	\item developed and maintained test stations for testing handheld devices involving highly accelerated lifecycle testing and failure mode analysis
    \item debugged board level electrical errors along with identifying hardware troubleshooting
    \item performed electrical and mechanical testing of all components within mobile device and was involved in troubleshooting and reporting the results
    \item qualified supplier manufacturing sites through statistical analysis
\end{innerlist}
\end{outerlist}

\blankline

%{\textbf{University of British Columbia, Mechanical Engineering}}, BC, Canada
%\begin{outerlist}
%	\item[] \textit{Complex Fluids Lab}
%	\hfill  \textbf{October 2010 to April 2011}
%\begin{innerlist}
%	\item designed and machined an experimental apparatus for testing of two non-Newtonian fluids
%	\item set up an electromechanical system in order to capture interactions between fluids, and automated data acquisition for further analysis.
%\end{innerlist}
%\end{outerlist}

\blankline

%{\textbf{Canfor Pulp and Paper Research and Development Coop}}, BC, Canada
%\begin{outerlist}
%	\item[] \textit{Research and Development Lab Technician}
%	\hfill  \textbf{January 2010 to April 2010}
%\begin{innerlist}
%	\item conducted analysis of testing destructive and non-destructive handsheet properties, fibre properties such as fibre length and coarseness, elasticity, viscosity and conductivity
% \item  monitored the pulp properties from Canfor's pulp lines by simulating the paper-making process,  and testing according to TAPPI standard procedures.
%    \item worked to automate the data entry process along with experimental testing automation
%\end{innerlist}
%\end{outerlist}


\section{Projects}
\begin{innerlist}
\item \href{http://www.tellmeabout.coffee}{coffee information website}
\item \href{http://youtu.be/ek3iVSD02Gw}{automatic wireless irrigation system} based on moisture sensor readings
\item \href{http://youtu.be/H9JlHjWq-NA}{autonomous tape following race car}
\item \href{http://youtu.be/ddSh_XTtdA0}{automated bicycle turn signal and brake light system}
\end{innerlist}

\section{Technical Details}
{\textbf{Database}}
\begin{innerlist}
\item extensive experience in scraping data from websites and parsing the data to be stored within a SQL or noSQL database
\item design and maintaining of PostgreSQL databases, including optimisation and administration roles
\item very comfortable using raw SQL to query data, or using an ORM wrapper
\item involved with database design using event triggers, jsonb datatypes, GIN indexing, materialized views, upserting and foreign data wrappers
\end{innerlist}

\blankline

{\textbf{Web Applications}}
\begin{outerlist}
\item[] \textit{Backend}
\begin{innerlist}
\item built multiple web applications using the Flask framework, along with appropriate framework extensions mainly for internal admin usage
\item packages often used are SQLAlchemy, pytest, gunicorn, flask-admin, WTForms
\item familiarity with a more full featured framework as well, having used Django to build a podcast hosting web application
\item spent time building REST APIs to serve to frontend frameworks to display visualisations like customised internal analytics dashboard and reporting
\item test driven development in practice whenever feasible, which isn't always the case
\item data warehousing from multiple feeds, with experience using pandas and numpy to manipulate data
\end{innerlist}
\item[] \textit{Frontend}
\begin{innerlist}
\item interactive visualisations using d3.js, customised to the user's liking through javascript functions
\item familiarity with manipulating HTML and CSS elements without having to use a library
\item aided in development of a React.js web application to handle customer sign ups and display tailored usage charts
\item very comfortable with using jQuery libraries and navigating DOM elements
\item dabbled in user experience design and page layouts
\end{innerlist}
\end{outerlist}

\blankline

{\textbf{Devops}}
\begin{innerlist}
\item maintaining CI/CD pipeline to ensure sane and reproducible work, through Chef and Jenkins
\item usage of Docker and containerized systems to keep infrastructure abstracted properly
\item set up of HTTP networking and security through nginx
\item experience with cloud/IaaS, namely AWS, Google Cloud, Digital Ocean, and Heroku
\item build, automate, update and maintain infrastructure e.g. databases, redis, internal services, etc
\end{innerlist}

\blankline

{\textbf{Data science}}
\begin{innerlist}
\item experienced in designing business driven data science platforms and applied AI solutions
\item able to design experiments and conduct validation on hypothesis through optimisation of metrics
\item developed predictive and explanatory models and machine learning algorithms to predict electricity demand and forecast consumption
\item analysed data from various sources and propose insights and correlations that can be derived from the data
\item always up to date with the most recent developments in machine learning by constantly reading papers and trying new algorithms
\end{innerlist}

\end{document}
%%%%%%%%%%%%%%%%%%%%%%%%%% End CV Document %%%%%%%%%%%%%%%%%%%%%%%%%%%%%
